\documentclass[11pt,english,letterpaper,oneside]{article}
\usepackage{amsmath,amssymb,amsfonts,amsthm,enumerate,enumitem}
\usepackage{mathpazo}
\usepackage[margin=2.5cm]{geometry}
\usepackage{graphicx,microtype}
\usepackage{textcomp}
\usepackage[T1]{fontenc}
\DisableLigatures[f]{encoding=T1}
\usepackage[skip=2pt]{caption}
\usepackage{hyperref}
\usepackage{setspace,array,float}
\usepackage{bm,upgreek}
\usepackage[compact]{titlesec}
\usepackage[none]{hyphenat}
\setlength{\parskip}{0.25cm}
\setlength{\parindent}{0cm}
\usepackage{natbib}
\setcitestyle{citesep={;},aysep={}}
\usepackage{babel}
\usepackage{datetime}
\frenchspacing
\usepackage[table]{xcolor}
\setcounter{secnumdepth}{2}
\setcounter{tocdepth}{2}

\makeatletter
\g@addto@macro\normalsize{%
  \setlength\belowdisplayskip{15pt}
  \setlength\belowdisplayshortskip{10pt}
}
\makeatother

\renewcommand{\familydefault}{ppl}

\renewcommand\theequation{S\arabic{equation}}

\newcommand{\ppgtk}{\textit{PPGtk}}
\newcommand{\code}[1]{\hspace{15pt} \texttt{#1}}

%%%%%%%%%%%%%%%%%%%%%%%%%%%%%%%%%%%%%%%%%%%%%%%%%

\begin{document}

\pagestyle{empty}


	{\huge Notes on PPGtk}
	\vspace{0.2in}
	
	Paul Blischak
	
	E-mail: \href{mailto:blischak.4@osu.edu}{blischak.4@osu.edu}
	\vspace{0.2in}
		
\tableofcontents

\newpage

\clearpage
\pagestyle{plain}
\setcounter{page}{1}



\section{Allele frequency estimation}


\subsection{The likelihood}

The likelihood of an individuals' read data given the population allele frequency can be computed by summing over the possible genotypes:

\begin{equation}
\mathcal{L}_i(p) = P(r_i|p) = \sum_{j = 0}^{m_i} P(r_i|j)P(j|p),
\end{equation}

\begin{equation*}
\text{where} \quad P(r_i|j) = \binom{t_i}{r_i}[\mathcal{G}_\epsilon(j)]^{r_i} [1 - \mathcal{G}_\epsilon(j)]^{(t_i - r_i)}, \quad \text{and} \quad P(j|p) = \binom{m_i}{j} p^j (1 - p)^{m_i - j}.
\end{equation*}

For multiple samples, we take the product of the individual likelihoods:

\begin{equation}
\mathcal{L}(p) = \prod_i \mathcal{L}_i(p) = \prod_i \left(\sum_{j = 0}^{m_i} P(r_i|j)P(j|p)\right).
\end{equation}

Taking the natural log gives us the log likelihood of the population allele frequency at a single site:

\begin{equation}
\ell(p) = \log \mathcal{L}(p) = \sum_i \log \left( \sum_{j = 0}^{m_i} P(r_i|j)P(j|p)\right).
\end{equation}

\subsection{Metropolis-Hastings algorithm}

\begin{equation}
P(p|r) \propto P(r|p)P(p)
\end{equation}

\begin{equation}
\alpha = \text{min} \left\{1, \frac{P(r|p^*)P(p^*)}{P(r|p)P(p)} \right\}
\end{equation}

\section{Inbreeding}

\subsection{The likelihood}

\subsection{Metropolis-Hastings algorithm}



\end{document}